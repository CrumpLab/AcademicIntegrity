\documentclass[]{book}
\usepackage{lmodern}
\usepackage{amssymb,amsmath}
\usepackage{ifxetex,ifluatex}
\usepackage{fixltx2e} % provides \textsubscript
\ifnum 0\ifxetex 1\fi\ifluatex 1\fi=0 % if pdftex
  \usepackage[T1]{fontenc}
  \usepackage[utf8]{inputenc}
\else % if luatex or xelatex
  \ifxetex
    \usepackage{mathspec}
  \else
    \usepackage{fontspec}
  \fi
  \defaultfontfeatures{Ligatures=TeX,Scale=MatchLowercase}
\fi
% use upquote if available, for straight quotes in verbatim environments
\IfFileExists{upquote.sty}{\usepackage{upquote}}{}
% use microtype if available
\IfFileExists{microtype.sty}{%
\usepackage{microtype}
\UseMicrotypeSet[protrusion]{basicmath} % disable protrusion for tt fonts
}{}
\usepackage[margin=1in]{geometry}
\usepackage{hyperref}
\hypersetup{unicode=true,
            pdftitle={Academic Integrity Resources: Department of Psychology Brooklyn College},
            pdfauthor={Matthew Crump},
            pdfborder={0 0 0},
            breaklinks=true}
\urlstyle{same}  % don't use monospace font for urls
\usepackage{natbib}
\bibliographystyle{apalike}
\usepackage{longtable,booktabs}
\usepackage{graphicx,grffile}
\makeatletter
\def\maxwidth{\ifdim\Gin@nat@width>\linewidth\linewidth\else\Gin@nat@width\fi}
\def\maxheight{\ifdim\Gin@nat@height>\textheight\textheight\else\Gin@nat@height\fi}
\makeatother
% Scale images if necessary, so that they will not overflow the page
% margins by default, and it is still possible to overwrite the defaults
% using explicit options in \includegraphics[width, height, ...]{}
\setkeys{Gin}{width=\maxwidth,height=\maxheight,keepaspectratio}
\IfFileExists{parskip.sty}{%
\usepackage{parskip}
}{% else
\setlength{\parindent}{0pt}
\setlength{\parskip}{6pt plus 2pt minus 1pt}
}
\setlength{\emergencystretch}{3em}  % prevent overfull lines
\providecommand{\tightlist}{%
  \setlength{\itemsep}{0pt}\setlength{\parskip}{0pt}}
\setcounter{secnumdepth}{5}
% Redefines (sub)paragraphs to behave more like sections
\ifx\paragraph\undefined\else
\let\oldparagraph\paragraph
\renewcommand{\paragraph}[1]{\oldparagraph{#1}\mbox{}}
\fi
\ifx\subparagraph\undefined\else
\let\oldsubparagraph\subparagraph
\renewcommand{\subparagraph}[1]{\oldsubparagraph{#1}\mbox{}}
\fi

%%% Use protect on footnotes to avoid problems with footnotes in titles
\let\rmarkdownfootnote\footnote%
\def\footnote{\protect\rmarkdownfootnote}

%%% Change title format to be more compact
\usepackage{titling}

% Create subtitle command for use in maketitle
\newcommand{\subtitle}[1]{
  \posttitle{
    \begin{center}\large#1\end{center}
    }
}

\setlength{\droptitle}{-2em}
  \title{Academic Integrity Resources: Department of Psychology Brooklyn College}
  \pretitle{\vspace{\droptitle}\centering\huge}
  \posttitle{\par}
  \author{Matthew Crump}
  \preauthor{\centering\large\emph}
  \postauthor{\par}
  \predate{\centering\large\emph}
  \postdate{\par}
  \date{2018-03-21}

\usepackage{booktabs}
\usepackage{amsthm}
\makeatletter
\def\thm@space@setup{%
  \thm@preskip=8pt plus 2pt minus 4pt
  \thm@postskip=\thm@preskip
}
\makeatother

\usepackage{amsthm}
\newtheorem{theorem}{Theorem}[chapter]
\newtheorem{lemma}{Lemma}[chapter]
\theoremstyle{definition}
\newtheorem{definition}{Definition}[chapter]
\newtheorem{corollary}{Corollary}[chapter]
\newtheorem{proposition}{Proposition}[chapter]
\theoremstyle{definition}
\newtheorem{example}{Example}[chapter]
\theoremstyle{definition}
\newtheorem{exercise}{Exercise}[chapter]
\theoremstyle{remark}
\newtheorem*{remark}{Remark}
\newtheorem*{solution}{Solution}
\begin{document}
\maketitle

{
\setcounter{tocdepth}{1}
\tableofcontents
}
\chapter*{Resource Documents}\label{resource-documents}
\addcontentsline{toc}{chapter}{Resource Documents}

\begin{itemize}
\tightlist
\item
  \href{Resources/Faculty\%20Action\%20Report.pdf}{Faculty Action
  Report}
\item
  \href{Resources/Academic_Integrity_Policy.pdf}{CUNY Academic Integrity
  Policy}
\item
  \href{Resources/F16_FacultyObligations_andSyllabus.pdf}{Faculty
  Obligations for Syllabus Preparation}
\item
  \href{Resources/110901_TermPapers_Sale.pdf}{Selling Term Papers}
\end{itemize}

\section{Contributing to this
resource}\label{contributing-to-this-resource}

The source code for compiling this document is in this github
repository: \url{https://github.com/CrumpLab/AcademicIntegrity}

You can contribute to this resource directly by suggesting edits to the
.Rmd files and submitting pull requests. You can also use the ``issues''
tab to leave comments and have discussions.

\chapter{Outline of Brookyln College
guidelines:}\label{outline-of-brookyln-college-guidelines}

In brief, there are two important obligations: to inform students that
they need to be aware of the CUNY Academic Integrity policy, and to
report suspected violations. These two obligations are described more
fully below:

\section{Syllabus Requirements}\label{syllabus-requirements}

According to the Brooklyn College ``Fall 2016 Obligations of the Faculty
and Syllabus Preparation'' document, all syllabi must include the
following statement concerning Academic Integrity:

``The faculty and administration of Brooklyn College support an
environment free from cheating and plagiarism. Each student is
responsible for being aware of what constitutes cheating and plagiarism
and for avoiding both. The complete text of the CUNY Academic Integrity
Policy and the Brooklyn College procedure for policy implementation can
be found at www.brooklyn.cuny.edu/bc/policies. If a faculty member
suspects a violation of academic integrity and, upon investigation,
confirms that violation, or if the student admits the violation, the
faculty member must report the violation.''

\section{Reporting Requirements}\label{reporting-requirements}

When a suspected violation is confirmed through investigation, or if the
student admits the violation, then faculty must report the violation.
The reporting process currently involves filing the FAR report (Faculty
Action Report for Incidents of Academic Dishonesty.) with the College's
Academic Integrity Officer, Mr.~Patrick Kavanagh (Office of the
Associate Provost for Academic Programs: 3208 Boylan Hall,
718-951-5771).

Patrick's email is:
\href{mailto:kavanagh@brooklyn.cuny.edu}{\nolinkurl{kavanagh@brooklyn.cuny.edu}}

The language in the FAR report suggests a broader obligation to file
this report whenever ``a faculty member finds compelling evidence that a
suspected violation of the CUNY policy on Academic Integrity has
occurred''.

\section{Reporting a suspected violation (FAR
REPORT)}\label{reporting-a-suspected-violation-far-report}

If you have questions about the process, Patrick is very helpful and
knowledgeable and can guide you through the details. The FAR report is
mostly self-explanatory and involves three sections

\subsection{Section A: Information about the
student}\label{section-a-information-about-the-student}

\begin{itemize}
\tightlist
\item
  fill out basic information about student name, CUNYFirst ID, email,
  date of incident etc.
\end{itemize}

\begin{enumerate}
\def\labelenumi{\alph{enumi}.}
\tightlist
\item
  You are given space to ``describe the incident''. You can attach an
  additional document if necessary
\item
  You are required to ``Indicate the outcome of your discussion of the
  violation with the student''. So, it is necessary to schedule a time
  to meet with or speak with this student.
\item
  Section A notes that: ``Evidence must accompany allegations. For
  plagiarism, please attach the syllabus and the assignment sheet in
  addition to the documentation of plagiarism, such as the Safe Assign
  Report or copy of the paper with portions found to be plagiarized
  highlighted along with the original source.''
\end{enumerate}

\subsection{Section B: Academic (Grade)
Sanction}\label{section-b-academic-grade-sanction}

\begin{itemize}
\tightlist
\item
  given the option to recommend 1) failing grade on Exam/Assignment, 2)
  Failing grade for Course, or Other (please explain)
\item
  required to report whether the student accepts the sanction
\end{itemize}

\subsection{Section C: Disciplinary Complaint
Option}\label{section-c-disciplinary-complaint-option}

\begin{itemize}
\tightlist
\item
  speak to Patrick about this one.
\end{itemize}

\section{Helpful links:}\label{helpful-links}

\url{http://libguides.brooklyn.cuny.edu/plagiarism}

\section{List of websites that students may use for
cheating:}\label{list-of-websites-that-students-may-use-for-cheating}

Most of these websites offer a similar set of services.

\begin{itemize}
\tightlist
\item
  \url{https://www.studyblue.com}
\item
  \url{https://www.coursehero.com}
\item
  \url{https://www.chegg.com}
\item
  \url{https://studentshare.net}
\item
  Brooklyn College Facebook in the know (a facebook group)
\item
  \url{https://www.slader.com}
\end{itemize}

There are many more similar kinds of sites

\chapter{Websites and cheating}\label{websites-and-cheating}

Students are sharing exams, assignments, old papers, homework questions,
etc. across numerous online platforms.

\begin{itemize}
\tightlist
\item
  \url{https://www.studyblue.com}
\item
  \url{https://www.coursehero.com}
\item
  \url{https://www.chegg.com}
\item
  \url{https://studentshare.net}
\item
  Brooklyn College Facebook in the know (a facebook group)
\item
  \url{https://www.slader.com}
\end{itemize}

\section{Exam/Assignment
Repositories}\label{examassignment-repositories}

These sites offer large repositories of test banks, assignments,
homework questions, papers, essays, answers etc., that can be downloaded
for free, or for the cost of a membership. All of the materials have
been uploaded by other students. You may be surprised to see most of if
not all of your own assignments/exams etc. uploaded to these websites.

If you find your materials on one of these websites, and do not want
your material there, then you may be able to contact have the material
removed by making a takedown request.

\section{\texorpdfstring{``Tutoring'' / paying
somebody}{Tutoring / paying somebody}}\label{tutoring-paying-somebody}

Most of theses sites offer personal tutoring services. For a fee, a
student subscriber can post a question to the tutoring service, and then
pay someone to ``help'' them with the question.

Many students use this to pay someone else to do their work for them,
and then they cheat by handing in the tutor's work as their own.
Remarkably, many of these sites make the results of tutoring requests
searchable. So, for example, if you had an account with the service, you
could search your own assignment questions to determine whether a tutor
has completed one of them. Then you could cross-reference the tutor
generated work with the work submitted by your students.

\section{Are your materials online?}\label{are-your-materials-online}

Find out easily by googling your portions of your materials. You can
also find yourself as an instructor on many of these sites, and then
find materials that students have uploaded and associated specifically
with your course.

\chapter{Preventative Measures}\label{preventative-measures}

The following are some suggestions for preventing academic integrity
issues before they start.

\section{Make your syllabus clear}\label{make-your-syllabus-clear}

\begin{enumerate}
\def\labelenumi{\arabic{enumi}.}
\item
  Your syllabus must include the previously described language from the
  \href{\%22Resources/F16_FacultyObligations_andSyllabus.pdf\%22}{obligations
  of the faculty document}
\item
  The mandatory language above does \textbf{not} prescribe how you, as
  the instructor, will sanction students for cheating/plagiarism. My
  opinion is that student expectations would be better managed by
  including specific langauge about how cheating/plagiarism will be
  handled in your course. This could be similar to a section about how
  late or missed assignments are dealt with. Here is some example
  language that I am developing for Experimental Psychology.
\end{enumerate}

\subsection{Cheating and Plagiarism in this
course}\label{cheating-and-plagiarism-in-this-course}

This class has a zero tolerance policy for cheating/plagiarism. If you
are caught cheating and/or plagiarizing then you will receive a failing
grade for this course, and a Faculty Action Report will be filed against
you.

You should assume that unless explictly told otherwise, that all
assignments in this course are to be completed by you alone. All of your
written work should be your own. Specifically, the following assignments
and papers are to be completed by you alone: list assignments here. The
exceptions to this rule include any group work, such as the following:
list group work here.

By handing in each assignment you are pledging your honor that you did
not violate the CUNY policy on Acamedic Integrity, which you are
responsible for reading and understanding.

All of the following are examples of cheating/plagiarism, that if
detected will result in serious academic sanctions:

\begin{enumerate}
\def\labelenumi{\arabic{enumi}.}
\tightlist
\item
  Copying work from another student and handing it in as your own.
\end{enumerate}

\begin{itemize}
\tightlist
\item
  including, copying a few sentences from someone else
\item
  including, paraphrasing parts of someone else's work
\item
  including, working together on assignments that are individual
  assignments
\end{itemize}

\begin{enumerate}
\def\labelenumi{\arabic{enumi}.}
\setcounter{enumi}{1}
\tightlist
\item
  Aiding cheating/plagiarism by sharing your work or giving your work to
  other students
\item
  Copying from a website
\item
  Having an online tutor, or other person, complete an assignment for
  you, and handing in that work as your own
\item
  Paraphrasing someone elses writing (e.g., from a website, from a
  textbook, from an article, from another student etc.) by substituting
  words and making changes and handing this work in as your own.
\item
  Self-plagiarising, including handing in a previous assignment from
  another class as part of completing an assignment for this class
\item
  All of the examples listed in the CUNY policy on Academic Integrity
\end{enumerate}

In this class, when an investigation of a suspected violation of the
Academic integrity policy demonstrates that a student knowingly and
deliberately violated the policy, the student will receive a failing
grade in this course. Following Brookyln College policy, any time an
instructor detects compelling evidence of a suspected academic
integrity, they must report it using the Faculty Action Report. When
compelling evidence of a suspected violation occurs in this course, the
following steps will be taken:

\begin{enumerate}
\def\labelenumi{\arabic{enumi}.}
\tightlist
\item
  You will be notified by me that I have detected compelling evidence of
  cheating/plagiarism
\item
  I will schedule a time to meet with you
\item
  I will fill out and submit a Faculty Action report, which describes
  the evidence, the result of our discussion, and whether you accept the
  academic sanction
\item
  \textbf{The academic sanction will be that you receive a failing grade
  for this course}
\item
  The FAR report (Faculty Action Report) will be filed with the
  College's Academic Integrity Officer Mr.~Patrick Kavanagh (Office of
  the Associate Provost for Academic Programs: 3208 Boylan Hall,
  718.951.5771 fax: 718.951.4559).
\item
  All students who have a FAR report filed against them can file an
  appeal, and should contact Mr.~Patrick Kavanagh to arrange for an
  appeal.
\end{enumerate}

\section{Honor Pledges}\label{honor-pledges}

Some students may be unaware that their actions constitute cheating and
plagiarism, additionally some students who violate academic integrity
rules may claim after the fact that they were unaware that their
behavior constituted cheating and/or plagiarism. Consider instituting an
honor pledge on exams and assignemts to make sure that students are
aware prior to completing the assignment. An example honor pledge that
can be placed at the beginning of an exam or assignment is:

``I pledge my honor that I am submitting my own work, that I did not
receive or give help on this individual assignment, and that I did not
violate the CUNY Academic integrity rules. I understand the consequences
for violating the CUNY academic integrity rules as described in the
syllabus for this course. Sign here: \_\_\_\_\_\_\_\_\_\_\_\_\_''

\section{Talk about cheating and plagiarism in
class}\label{talk-about-cheating-and-plagiarism-in-class}

Consider discussing cheating and plagiarism and how you will deal with
these issues in your class during class time.

\section{Turnitin and SafeAssign}\label{turnitin-and-safeassign}

The blackboard system offers two services for detecting plagiarism in
written work, these are turnitin and SafeAssign. You can enable
SafeAssign and Turnitin on blackboard, and have students submit written
work on Blackboard. Both services check the pool of submitted documents
for verbatim similarities. They do not detect paraphrasing. SafeAssign
checks similarities between the submitted documents for the assignment,
whereas Turnitin addditionall compares the submitted documents to
websites, and previously submitted documents in their database.

Neither service works perfectly, or even near perfectly. For example, a
students paper might return a very high similarity score (\textgreater{}
90\%). This could mean that 90\% of their document was lifted verbatim
from another single sourcce (obvious cheating), or it could be a sum
total of small partial matches across many different documents. A
student's paper could also return a very low similarity score
(\textless{}10\%), consistent with no cheating. However, further
scrutiny of the reason for the match might show that one or two entire
sentences have been directly copied verbatim from another students or
source.

Instructors have the option of letting students pre-submit their work to
determine their safeassign or turnitin similarity score prior to
submitting their work. This may be useful in preventing cheating and
plagiarism.

\bibliography{book.bib,packages.bib}


\end{document}
